% created by Joanna Plaskonka (joanna.plaskonka@nsn.com)
\documentclass[english,handout]{beamer}
%\documentclass[english]{beamer}

\definecolor{nokia_blue}{RGB}{18,65,145}
\definecolor{nokia_light_blue}{RGB}{0,201,255}
\definecolor{nokia_dark_grey}{RGB}{104,113,122}
\definecolor{nokia_medium_grey}{RGB}{168,187,192}
\definecolor{nokia_light_grey}{RGB}{216,217,218}

\mode<presentation>
{
	\usetheme{Boadilla}
	\usecolortheme{dove}
	\setbeamercovered{transparent}
}

\usepackage[utf8]{inputenc}
\usepackage[T1]{fontenc}
\usepackage{babel}
\usepackage{amsmath}
\usepackage{amsfonts}

\usepackage{verbatim}

\usepackage{hyperref} % for linking web pages

\usepackage{graphicx}  % for graphics

\usepackage{times}
\usepackage{pgf}
\usepackage{multimedia}
\usepackage{color}

%\usepackage{pdfpages} % for inserting a pdf file in LaTeX document
\usepackage{verbatim} % will be used for multiline comments

\definecolor{lightgrey}{rgb}{0.9,0.9,0.9}
\definecolor{darkgreen}{rgb}{0,0.6,0}

\usepackage{ragged2e}

\usepackage{epstopdf} % for convertion eps --> pdf

\setbeamercolor{normal text}{fg=nokia_blue,bg=white}
\setbeamercolor{alerted text}{fg=red}
\setbeamercolor{example text}{fg=green,bg=white}
\setbeamercolor{structure}{fg=nokia_blue}
\setbeamercolor{palette primary}{fg=white,bg=nokia_blue} % changed this
\setbeamercolor{palette secondary}{fg=white,bg=nokia_blue} % changed this
\setbeamercolor{palette tertiary}{fg=white,bg=nokia_blue} % changed this
\setbeamercolor{block title}{bg=nokia_blue, fg=white}
\setbeamercolor{upper separation line head}{bg=nokia_medium_grey}
\setbeamercolor{lower separation line head}{bg=nokia_medium_grey}
\setbeamercolor{background canvas}{bg=white}

\beamertemplatenavigationsymbolsempty % suppress the navigation bar
\renewcommand{\arraystretch}{1.5}
\setlength{\arrayrulewidth}{1pt}

\newenvironment{narrowblock}[1]{%
\begin{center}
\begin{minipage}{10.5cm}
\begin{block}{#1}
}{%
\end{block}
\end{minipage}
\end{center}
}

\title{Refactoring Workshop}
\subtitle{Nokia Academy 2016}
\author{Bogumił Chojnowski} 
\institute[Nokia]{\textbf{Nokia}}
\date[Nov 29, 2016]{Wrocław, November 29}
\logo{\includegraphics[height=1cm]{NOKIA_LOGO_RGB_HR.png} \vspace{-.2cm}}

\AtBeginSection[]{
\begin{frame}
	\frametitle{Agenda}
	\tableofcontents[currentsection]
\end{frame}
}

%\AtBeginSubsection[] % This will display the list of contents before the beginning of each subsection
%{
%\begin{frame}
%      \frametitle{Agenda}
%      \tableofcontents[currentsubsection]
%\end{frame}
%}

\hypersetup{pdfstartview={Fit}}

\begin{document}


\begin{frame}[fragile]
    \titlepage
\end{frame}


\begin{frame}
\frametitle{Agenda}
	\tableofcontents
\end{frame}

\section{Pre-Test}
\begin{frame}
\frametitle{Pre-Test}
\end{frame}

\section{Introduction}

\begin{frame}
\frametitle{The Refactoring}
\begin{narrowblock}{Quote for Tuesday}

Refactoring is a controlled technique for improving the design of an existing code base. Its essence is applying a series of small behavior-preserving transformations, each of which "too~small to be worth doing".\footnote{{\url{http://www.martinfowler.com/books/refactoring.html}}}
\end{narrowblock}
\end{frame}

\subsection{The Motivation}

\begin{frame}
\frametitle{The Motivation}
\begin{itemize}
\item<1-> Code Smells
\item<2-> Detect and fix bugs
\item<3-> Preparation for introducing new feature
\end{itemize}
\end{frame}

\subsection{The Goal}

\begin{frame}
\frametitle{The Goal}
\begin{itemize}[<+->]
\item Understandability
\item Correctness
\item Mantainability
\item Extensibility
\item Decoupling
\end{itemize}
\end{frame}

\subsection{The Principles}

\begin{frame}
\frametitle{The Principles}
\begin{itemize}[<+->]
\item KISS 
\item DRY (Abstraction Principle)
\item YAGNI
\item SOLID
\item GRASP
%\item Law of Demeter
%\item Tell, don't ask
%\item Separation of Concerns
%\item Design by Contract
\end{itemize}
\end{frame}

\subsection{The Discipline}

\begin{frame}
\frametitle{The Discipline}
\begin{itemize}[<+->]
\item Small changes -- too small to be worth doing
\item Keep it green
\item Commit often
\item Expect to fail even more often (dead ends)
\item \textbf{Do never change tests and code simultaneously!}
\end{itemize}
\end{frame}

\begin{frame}
\frametitle{The Discipline}
\begin{center}
\includegraphics[width=12cm]{img/red-green-refactor.pdf}
\end{center}
\end{frame}

%\subsection{The Tools}
%
%\begin{frame}
%\frametitle{The Tools}
%\begin{itemize}[<+->]
%\item Extract Subroutin
%\item Extract Class
%\item Fight Primitives
%\item Extract Logic
%\item Interface Segregation
%\end{itemize}
%\end{frame}

\section{Workshop}

\begin{frame}[fragile]
\frametitle{Introduction}
%git remote add refactoring-workshop \url{https://github.com/bodzio528/nokia-academy-2016.git}
\begin{verbatim}
git checkout master
cd RefactoringWorkshop
qtcreator CMakeLists.txt
\end{verbatim}
\end{frame}


\subsection{Extract Subroutines}

\begin{frame}
\frametitle{Task \thesubsection: Extract Subroutines}

\begin{narrowblock}{Client's Story}
I want receiving Pause Indication to disable handling of timeout events.
\end{narrowblock}

\pause
\only<1-3>{
\begin{narrowblock}{Problems}
\begin{itemize}
\item<2-> WET (in fact, massive duplications)
\item<3-> Overengineering (try-catch to message dispatch, really?)
\end{itemize}
\end{narrowblock}
}

\only<4->{
\begin{narrowblock}{Hints}
\begin{enumerate}
\item<4-> Split SnakeController::receive method into manageable pieces.
\item<5-> Reenable tests.
\item<6-> Add implementation.
\end{enumerate}
\end{narrowblock}
}

\end{frame}

\subsection{Extract Subclasses}

\begin{frame}
\frametitle{Task \thesubsection: Extract Subclasses}

\begin{narrowblock}{Client's Story}
\textbf{BUG!} SnakeController places new food at invalid position! Controller should send another FoodReq in such case.
\end{narrowblock}

\pause
\begin{narrowblock}{Hints}
\begin{itemize}[<+->]
\item This issue is related only to gameboard dimension.
\item SOLID: Single Responsibility.
\end{itemize}
\end{narrowblock}
\end{frame}

\subsection{Fight Primitives}

\begin{frame}
\frametitle{Task \thesubsection: Fight Primitives}

\begin{narrowblock}{Client's Story}
Our beloved collegues from Moscow have deployed new, shiny Position and Dimension classes that have some nice features. More important thing -- those are designated as incoming standard of in/out SnakeController communication! 
\end{narrowblock}

\pause
\begin{narrowblock}{Hints}
\begin{itemize}[<+->]
\item Scalar values are prone to errors and meaningless on their own. 
\item Fighting primitives is one step towards semantic programming.
\item Expressing programmer thought in domain language is better than obscured bitshift calculations.
\end{itemize}
\end{narrowblock}
\end{frame}

\subsection{Extract Logic}

\begin{frame}
\frametitle{Task \thesubsection: Extract Logic}

\begin{narrowblock}{Client's Story}
I want to receive Score Indication with value of current snake length!
\end{narrowblock}

\pause
\begin{narrowblock}{Hints}
\begin{itemize}[<+->]
\item This is related only to snake internal representation.
\item Tell, do not ask!
\item GRASP: Information Expert.
\end{itemize}
\end{narrowblock}
\end{frame}

\subsection{Segregate Interfaces}

\begin{frame}
\frametitle{Task \thesubsection: Segregate Interfaces}

\begin{narrowblock}{Client's Story}
Simon says: Torus World.
\end{narrowblock}

\pause
\begin{narrowblock}{Hints}
\begin{itemize}[<+->]
\item 'T' in config string instead of 'W' indicates this feature enabled.
\item New tests are initialy disabled, old tests remain valid.
\item SOLID: Interface Segregation.
\end{itemize}
\end{narrowblock}

\end{frame}

\section{Post-Test}
\begin{frame}
\frametitle{Post-Test}
\end{frame}

\begin{frame}
\begin{center}
Thank you for participating!
\end{center}
\end{frame}

\end{document}